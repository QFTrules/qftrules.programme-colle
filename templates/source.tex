\documentclass[a4paper,11pt]{article}
\usepackage[utf8]{inputenc}
\usepackage[T1]{fontenc}
\usepackage[french]{babel}
\usepackage{amsmath}
\usepackage{amssymb}
\usepackage{amsfonts}
\usepackage{multicol}
\usepackage{xcolor}
\usepackage{wasysym}
\input{/home/eb/Dropbox/texmf/tex/latex/Preambles/symbols.sty}
\usepackage[francais,bloc,completemulti]{automultiplechoice}

\begin{document}

% barème général
\baremeDefautM{b=2,m=-1}

% \title{Automultiplechoice Example}
% \author{Your Name}
% \date{\today}
% \maketitle

% \exemplaire{1}{
\begin{copieexamen}[1]
%%% debut de l’en-tête des copies :
\begin{minipage}{.4\linewidth}
\centering\large\bf Interrogation de cours\\ 01/01/2008\end{minipage}
\champnom{\fbox{
\begin{minipage}{.5\linewidth}
    \vspace*{.5cm}
    Nom et prénom :
\namefielddots
\vspace*{1mm}
\end{minipage}
}}
\begin{center}\em

Durée : 10 minutes.

Aucun document n’est autorisé.
L’usage de la calculatrice est interdit.
Les questions faisant apparaître le symbole \multiSymbole{} peuvent
présenter zéro, une ou plusieurs bonnes réponses. Les autres ont
une unique bonne réponse. Veillez à cocher correctement et lisiblement les cases, de la façon suivante : {\Large \rm \XBox}.
% Des points négatifs pourront être affectés à de \emph{très
% mauvaises} réponses.
\end{center}
\vspace{1ex}