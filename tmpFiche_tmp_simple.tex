%!TEX root = /home/eric/Dropbox/CPGE/Info/Devoirs/DS.Info/[1]/DS1_info.tex
%%% PREAMBLE %%%
% \input{DM.sty}%$
\renewcommand{\classe}{PCSI}

% \newcommand{\Coursun}{}
% \newcommand{\Exoun}{}
% \newcommand{\Coursdeux}{}
% \newcommand{\Exodeux}{}
% \newcommand{\Courstrois}{}
% \newcommand{\Exotrois}{}
% \newcommand{\classe}{}

%%% DOCUMENT %%%
\documentclass[a4paper,11pt]{book}
\usepackage[utf8]{inputenc}
\usepackage[T1]{fontenc}
\usepackage[french]{babel}
\usepackage[autolanguage]{numprint}
\usepackage{lmodern}
\usepackage{amsmath}
\usepackage{amssymb}
\usepackage{cases}
\usepackage{mathrsfs}
\usepackage{textcomp}
\usepackage[top=0cm,bottom=0cm,left=1cm,right=1cm]{geometry}
\usepackage{stmaryrd}
\usepackage{inputenc}
\usepackage{moreverb}
\usepackage{listings}
\usepackage{graphicx}
\usepackage{fancyvrb}
\usepackage{listings}
\usepackage{wrapfig}
\usepackage{subfigure}
\usepackage{pgfplots}
\usepackage{esint}
\usepackage{tikz}
\usepackage[europeanresistors]{circuitikz}
\usepackage{multicol}
\usepackage{tabularx}
\usepackage{lipsum}
\usepackage{chngcntr}
\usepackage{fancyhdr}
\usepackage{exercise}
\usepackage{titlesec}
\usepackage{titletoc}
\usepackage{vwcol}
\counterwithin{table}{section}
\usepackage[titletoc]{appendix}
\usepackage{changepage}
\usepackage{mdframed}
\usepackage{pdfpages}
\usepackage{enumitem}
\usepackage{multirow}
\pgfplotsset{compat=1.16}

%%% CAPTIONS %%%
%\DeclareCaptionLabelSeparator{trait}{$\,-\,$}
\usepackage[font      = small,
			%labelfont = bf,
			labelsep  = endash]{caption}

%%% SYMBOLS %%%
\input{$HOME/Dropbox/texmf/tex/latex/Preambles/symbols.sty}%$
\renewcommand{\classe}{PCSI}
\input{$HOME/Dropbox/texmf/tex/latex/Preambles/programme.tex}%$
\renewcommand{\classe}{PCSI}
%\newcommand{\vv}[1]{\Vec{#1}}

%%% COLORS %%%
\def\colorSec{coolblack}
%\def\colorTP{brickred}
\def\colorTP{coolblack}

% %%% PARAGRAPHS %%%
% \renewcommand{\theparagraph}{}
% \newcommand{\parag}[1]{
% \paragraph{{#1}~---$\!\!\!\!$}}

%%% PATH %%%
\def\pathtocolles{/home/eric/pCloudDrive/Colles/Exos.de.Colles}
\def\pathtoexos{$HOME/Dropbox/CPGE/Physique/Exercices/Recueil}%$
\makeatletter
\def\input@path{{\pathtoexos/Dimensionnelle/}{\pathtoexos/Optique/}{\pathtoexos/Mecanique/}{\pathtoexos/Thermo/}{\pathtoexos/Fluide/}{\pathtoexos/Electromag/}{\pathtoexos/Ondes/}{\pathtoexos/Quantique/}{\pathtocolles/Mecanique}{\pathtoexos/Figure/}{\pathtoexos/Mecanique/Figures/}{\pathtoexos/Thermo/Figures/}{\pathtoexos/Optique/Figures/}{\pathtoexos/Fluide/Figures/}{\pathtoexos/Electromag/Figures/}{\pathtoexos/Ondes/Figures/}{\pathtoexos/Quantique/Figures/}}
\makeatother
\graphicspath{{./Figures/}{\pathtoexos/Figure/}{\pathtoexos/Mecanique/Figures/}{\pathtoexos/Thermo/Figures/}{\pathtoexos/Optique/Figures/}{\pathtoexos/Fluide/Figures/}{\pathtoexos/Electromag/Figures/}{\pathtoexos/Ondes/Figures/}{\pathtoexos/Quantique/Figures/}{\pathtoexos/Info/Figures/}}


\rhead{}
\chead{}
\renewcommand{\headrulewidth}{-15pt}
\newcommand{\colleskip}{\vspace{0.3cm}}
\newcommand{\heig}{2cm}
\begin{document}


\newcommand{\Coursun}{Ponts diviseurs}
\newcommand{\Coursdeux}{Dipôles électrocinétiques}
\newcommand{\Courstrois}{Définitions et lois générales dans le cadre de l’ARQS}
\newcommand{\Exoun}{Transformation étoile-triangle}
\newcommand{\Exodeux}{Tension et intensité dans un circuit résistif simple}
\newcommand{\Exotrois}{Comparateur de tension}


\vspace{-2cm}
\begin{center}
\textsc{Fiche de colle}\\
Physique - \classe
\end{center}
\vspace{-0.9cm}
\begin{minipage}{0.48\textwidth}
\flushleft{Correcteur : Eric Brillaux
}
\end{minipage}\hfill
\begin{minipage}{0.48\textwidth}
\flushright{Date : 05/11/2024}\end{minipage}
\vspace{0.2cm}

%%% TABLE %%%
\centering
\begin{tabularx}{\textwidth}{|c|X|}
\hline
\multicolumn{2}{|l|}{\begin{minipage}{2cm}Nom :\hspace{1.2cm}\end{minipage}} \\[0.2cm]
\hline
\multirow{3}{*}{Note :\hspace{1.2cm}} & \textsc{Question de cours} \\[0.1cm]
 & Énoncé : \Coursun \\[0.1cm]
 & Commentaire : \\[\heig]
 \cline{2-2}
S :\hspace{1.2cm} & \textsc{Exercice} \\[0.1cm]
C :\hspace{1.2cm} & Énoncé : \Exoun \\[0.1cm]
A :\hspace{1.2cm} & Commentaire : \\[\heig]
\hline
\end{tabularx}
%%%%%%%%%%


\colleskip
%%% TABLE %%%
\centering
\begin{tabularx}{\textwidth}{|c|X|}
\hline
\multicolumn{2}{|l|}{\begin{minipage}{2cm}Nom :\hspace{1.2cm}\end{minipage}} \\[0.2cm]
\hline
\multirow{3}{*}{Note :\hspace{1.2cm}} & \textsc{Question de cours} \\[0.1cm]
 & Énoncé : \Coursdeux \\[0.1cm]
 & Commentaire : \\[\heig]
 \cline{2-2}
 S :\hspace{1.2cm} & \textsc{Exercice} \\[0.1cm]
 C :\hspace{1.2cm} & Énoncé : \Exodeux \\[0.1cm]
 A :\hspace{1.2cm} & Commentaire : \\[\heig]
\hline
\end{tabularx}
%%%%%%%%%%

\colleskip
%%% TABLE %%%
\centering
\begin{tabularx}{\textwidth}{|c|X|}
\hline
\multicolumn{2}{|l|}{\begin{minipage}{1.5cm}Nom :\hspace{1.2cm}\end{minipage}} \\[0.2cm]
\hline
\multirow{3}{*}{Note :\hspace{1.2cm}} & \textsc{Question de cours} \\[0.1cm]
 & Énoncé : \Courstrois \\[0.1cm]
 & Commentaire : \\[\heig]
 \cline{2-2}
 S :\hspace{1.2cm} & \textsc{Exercice} \\[0.1cm]
 C :\hspace{1.2cm} & Énoncé : \Exotrois \\[0.1cm]
 A :\hspace{1.2cm} & Commentaire : \\[\heig]
\hline
\end{tabularx}
%%%%%%%%%%

% % to double the page
% \newpage

% \renewcommand{\Coursun}{}
% \renewcommand{\Exoun}{}
% \renewcommand{\Coursdeux}{}
% \renewcommand{\Exodeux}{}
% \renewcommand{\Courstrois}{}
% \renewcommand{\Exotrois}{}

% \vspace{-2cm}
% \begin{center}
% \textsc{Fiche de colle}\\
% Physique - \classe
% \end{center}
% \vspace{-0.9cm}
% \begin{minipage}{0.48\textwidth}
% \flushleft{Correcteur : Eric Brillaux
% }
% \end{minipage}\hfill
% \begin{minipage}{0.48\textwidth}
\flushright{Date : 12/11/2024}% \end{minipage}
% \vspace{0.2cm}

% %%% TABLE %%%
% \centering
% \begin{tabularx}{\textwidth}{|c|X|}
% \hline
% \multicolumn{2}{|l|}{\begin{minipage}{2cm}Nom :\hspace{1.2cm}\end{minipage}} \\[0.2cm]
% \hline
% \multirow{3}{*}{Note :\hspace{1.2cm}} & \textsc{Question de cours} \\[0.1cm]
%  & Énoncé : \Coursun \\[0.1cm]
%  & Commentaire : \\[\heig]
%  \cline{2-2}
%  S :\hspace{1.2cm} & \textsc{Exercice} \\[0.1cm]
%  C :\hspace{1.2cm} & Énoncé : \Exoun \\[0.1cm]
%  A :\hspace{1.2cm} & Commentaire : \\[\heig]
% \hline
% \end{tabularx}
% %%%%%%%%%%


% \colleskip
% %%% TABLE %%%
% \centering
% \begin{tabularx}{\textwidth}{|c|X|}
% \hline
% \multicolumn{2}{|l|}{\begin{minipage}{2cm}Nom :\hspace{1.2cm}\end{minipage}} \\[0.2cm]
% \hline
% \multirow{3}{*}{Note :\hspace{1.2cm}} & \textsc{Question de cours} \\[0.1cm]
%  & Énoncé : \Coursdeux \\[0.1cm]
%  & Commentaire : \\[\heig]
%  \cline{2-2}
%  S :\hspace{1.2cm} & \textsc{Exercice} \\[0.1cm]
%  C :\hspace{1.2cm} & Énoncé : \Exodeux \\[0.1cm]
%  A :\hspace{1.2cm} & Commentaire : \\[\heig]
% \hline
% \end{tabularx}
% %%%%%%%%%%

% \colleskip
% %%% TABLE %%%
% \centering
% \begin{tabularx}{\textwidth}{|c|X|}
% \hline
% \multicolumn{2}{|l|}{\begin{minipage}{1.5cm}Nom :\hspace{1.2cm}\end{minipage}} \\[0.2cm]
% \hline
% \multirow{3}{*}{Note :\hspace{1.2cm}} & \textsc{Question de cours} \\[0.1cm]
%  & Énoncé : \Courstrois \\[0.1cm]
%  & Commentaire : \\[\heig]
%  \cline{2-2}
%  S :\hspace{1.2cm} & \textsc{Exercice} \\[0.1cm]
%  C :\hspace{1.2cm} & Énoncé : \Exotrois \\[0.1cm]
%  A :\hspace{1.2cm} & Commentaire : \\[\heig]
% \hline
% \end{tabularx}
% %%%%%%%%%%


\end{document}
